% Abstract ====================================================================
\pdfbookmark[1]{Abstract}{Abstract}
\chapter*{Abstract}

Several real-world classification problems are example-dependent cost-sensitive in nature, where the 
costs due to misclassification vary between examples and not only within classes. However, standard 
classification methods do not take these costs into account, and assume a constant cost of 
misclassification errors. This approach is not realistic in many real-world applications. For  
example in credit card fraud detection, failing to detect a fraudulent transaction may have an 
economical impact from a few to thousands of Euros, depending on the particular transaction and card 
holder. In churn modeling, a model is used for predicting which customers are more likely to 
abandon a service provider. In this context, failing to identify a   profitable or unprofitable 
churner has a significant different economic   result. Similarly, in direct marketing, wrongly 
predicting that a customer   will not accept an offer when in fact he will, may have different 
financial impact, as not all   customers generate the same profit. Lastly, in credit scoring, 
accepting   loans from bad customers does not have the same economical loss, since customers have 
different   credit lines, therefore, different profit.

Accordingly, the goal of this thesis is to provide an in-depth analysis of example-dependent 
cost-sensitive classification. The first part of this manuscript is dedicated to explain the 
particularities of the four real-world classification problems that are the focus of this thesis, in 
particular, credit card fraud detection, credit scoring, churn modeling and direct marketing. In 
general, we show why each of the applications is example-dependent cost-sensitive, and we elaborate 
a framework for the analysis of each problem. 

\todo{add part 1}

In the second part of this work, we introduce our previously proposed example-dependent 
cost-sensitive methods, namely, Bayes minimum risk, cost-sensitive logistic regression, 
cost-sensitive decision trees algorithm and ensemble of cost-sensitive decision trees.
Moreover, we present the library \mbox{\textit{CostCla}} that we develop as part of the thesis. 
This library is an open-source implementation of all the algorithms covered in this manuscript.


Finally, the results will show the importance of using the real example-dependent financial 
costs associated with real-world applications, since there are significant differences in the 
results when evaluating a model using a traditional cost-insensitive measure such as the accuracy 
or F1Score,  than when using the savings, leading to the conclusion of the importance of using the 
real practical financial costs of each context.


